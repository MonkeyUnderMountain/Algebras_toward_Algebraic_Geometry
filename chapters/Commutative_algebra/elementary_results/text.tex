\section{Some fundamental Results}

\Yang{To be completed}

\subsection{Rings and modules}

    In the appendix and all the note, the ``ring'' is always commutative and with identity.
    We denote by \(\Spec A\) the set of prime ideals of a ring \(A\).
    We denote by \(\mSpec A\) the set of maximal ideals of \(A\).
    Let \(I \subset A\) be an ideal of \(A\).
    We define 
    \[ V(I) \coloneqq \{ \frakp \in \Spec A \colon I \subset \frakp\}. \]

    Let \(\fraka,\frakb\) be ideals of \(A\).
    We define 
    \[ (\fraka : \frakb) \coloneqq \{ a \in A \colon a\frakb \subset \fraka\}. \]
    This is an ideal of \(A\).
    
    Let \(\rad(A)\) be the Jacobian radical of \(A\), i.e., the intersection of all maximal ideals of \(A\).
    Let \(\nil(A)\) be the nilradical of \(A\), i.e., the ideal of \(A\) consisting of all nilpotent elements.
    
    \begin{proposition}
        Let \(A\) be a ring.
        Then we have 
        \[ \nil(A) = \bigcap_{\frakp \in \Spec A} \frakp. \]
    \end{proposition}
    \begin{proof}
        \Yang{To be completed.}
    \end{proof}

    \begin{proposition}\label{prop: prime avoidance lemma primity of prime ideals}
        Let $A$ be a ring, $\frakp,\frakp_i$ prime ideals of $A$ and \(\fraka,\fraka_i\) ideals of $A$. 
        \begin{enumerate}
            \item Suppose \(\fraka \subset \bigcup_{i=1}^n \frakp_i\). 
            Then there exists \(i\) such that \(\fraka \subset \frakp_i\).
            \item Suppose \(\bigcap_{i=1}^n \fraka_i \subset \frakp\). 
            Then there exists \(i\) such that \(\fraka_i \subset \frakp\).
        \end{enumerate}
    \end{proposition}
    \begin{proof}
        \Yang{To be completed.}
    \end{proof}

    Let \(M\) be an \(A\)-module.
    We say that \(M\) is \emph{finite} if there exists an exact sequence
    \[ A^n \to M \to 0. \]
    We say that \(M\) is \emph{finite presented} if there exists an exact sequence
    \[ A^m \to A^n \to M \to 0. \]
    If \(A\) is a noetherian ring, then every finite \(A\)-module is finite presented.

    \begin{definition}\label{def: support of a module}
        Let \(A\) be a ring and \(M\) an \(A\)-module.
        The \emph{support} of \(M\) is defined as
        \[
            \Supp M \coloneqq \{ \frakp \in \Spec A \colon M_\frakp \neq 0\}.
        \]
    \end{definition}

    The \emph{annihilator} of \(M\) is defined as
    \[ \Ann M \coloneqq \{ a \in A \colon aM = 0\}. \]
    This is an ideal of \(A\).

    \begin{proposition}\label{prop: support of a finite module}
        Let \(A\) be a ring and \(M\) a finite \(A\)-module.
        Then \(\Supp M = V(\Ann M)\).
        In particular, \(\Supp M\) is a closed subset of \(\Spec A\).
    \end{proposition}
    \begin{proof}
        \Yang{To be completed.}
    \end{proof}

\subsection{Localization}

    \begin{definition}\label{def: localization}
        Let \(A\) be a ring and \(S \subset A\) a multiplicative subset, i.e., \(1 \in S\) and \(s_1,s_2 \in S\) implies \(s_1 s_2 \in S\).
        Let \(M\) be an \(A\)-module.
        The \emph{localization} of \(M\) at \(S\) is defined as
        \[ S^{-1}M \coloneqq M \times S / \sim, \]
        where \((m,s) \sim (n,t)\) if there exists \(u \in S\) such that \(u(tm - sn) = 0\).
        We denote the equivalence class of \((m,s)\) by \(\frac{m}{s}\) or \(m/s\).

        The localization \(S^{-1}A\) is still a ring and hence an \(A\)-algebra.
        The localization \(S^{-1}M\) is an \(S^{-1}A\)-module.
        If \(M=B\) is an \(A\)-algebra, then \(S^{-1}B\) is an \(S^{-1}A\)-algebra.
    \end{definition}

    \begin{example}\label{eg: localization at prime ideal and multiplicative set generated by one element}
        Let \(A\) be a ring, \(\frakp\) a prime ideal of \(A\) and \(M\) an \(A\)-module.
        Then \(S = A \setminus \frakp\) is a multiplicative subset.
        The localization \(S^{-1}M\) is denoted by \(M_\frakp\) and called the localization of \(M\) at \(\frakp\).

        Let \(f \in A\) be an element.
        Then \(S = \{ f^n \colon n \geq 0\}\) is a multiplicative subset.
        The localization \(S^{-1}M\) is denoted by \(M[1/f]\).
    \end{example}

    % \begin{proposition}\label{universal_property_of_localization}
    %     Let \(A\) be a ring, \(S \subset A\) a multiplicative subset and \(M\) an \(A\)-module.
    %     Then the localization \(S^{-1}M\) satisfies the following universal property:
    %     for any \(A\)-module \(N\) such that every element of \(S\) acts on \(N\) as an automorphism, there is a unique morphism of \(A\)-modules \(S^{-1}M \to N\) making the following diagram commute:
    %     \[
    %         \begin{tikzcd}
    %             M \arrow[r] \arrow[rd] & S^{-1}M \arrow[dashed]{d}{} \\
    %             & N
    %         \end{tikzcd}
    %     \]
    %     \Yang{To be completed.}
    % \end{proposition}

    \begin{proposition}\label{prop: when is localization injective}
        The natural map \(A \to S^{-1}A\) is injective if and only if \(S\) contains no zero divisors.
    \end{proposition}

    \begin{proposition}\label{prop:localization_and_tensor_product}
        Let \(A\) be a ring, \(S \subset A\) a multiplicative subset and \(M\) an \(A\)-module.
        Then we have a natural isomorphism of \(S^{-1}A\)-modules
        \[ S^{-1}M \cong M \otimes_A S^{-1}A. \]
    \end{proposition}

    \begin{proposition}\label{prop:localization_are_flat}
        The localization \(S^{-1}A\) is a flat \(A\)-algebra.
    \end{proposition}

\subsection{Chain conditions}

    \begin{definition}\label{def:noetherian_and_artinian_rings_and_modules}
        Let \(A\) be a ring.
        We say that \(A\) is \emph{noetherian} (resp. \emph{artinian}) if every ascending (resp. descending) chain of ideals of \(A\) stabilizes.
    \end{definition}

    \begin{proposition}\label{prop:equivalent_definitions_of_noetherian_rings}
        Let \(A\) be a ring.
        The following are equivalent:
        \begin{enumerate}
            \item \(A\) is noetherian.
            \item Every ideal of \(A\) is finitely generated.
            \item Every non-empty set of ideals of \(A\) has a maximal element (with respect to inclusion).
        \end{enumerate}
    \end{proposition}
    \begin{proof}
        \Yang{To be completed.}
    \end{proof}

    \begin{theorem}[Hilbert's Basis Theorem]\label{thm:Hilbert_basis_theorem}
        If \(A\) is a noetherian ring, then \(A[x]\) is noetherian.
    \end{theorem}
    \begin{proof}
        \Yang{To be completed.}
    \end{proof}

    \begin{remark}\label{rmk:Hilbert_basis_theorem_for_ring_of_power_series}
        By a similar argument replacing \(\deg f\) by \(\ord f\), we can show that if \(A\) is noetherian, then the formal power series ring \(A[[x]]\) is also noetherian.
    \end{remark}

\subsection{Nakayama's Lemma}

    \begin{theorem}[Nakayama's Lemma]\label{thm: Nakayama's lemma}
        Let $A$ be a ring and $\frakM$ be its Jacobi radical.
        Suppose $M$ is a finitely generated $A$-module.
        If $\fraka M = M$ for $\fraka \subset \frakM$, then $M = 0$.
    \end{theorem}
    \begin{proof}
        Suppose $M$ is generated by $x_1,\cdots,x_n$.
        Since $M = \fraka M$, formally we have $(x_1,\cdots,x_n)^T = \Phi (x_1,\cdots, x_n)^T$ for $\Phi \in M_n(\fraka)$.
        Then $(\Phi - \id) (x_1,\cdots, x_n)^T = 0$. 
        Note that $\det (\Phi - \id) = 1+a$ for $a \in \fraka \subset \frakM$.
        Then $\Phi-\id$ is invertible and then $M=0$.
    \end{proof}

    \begin{remark}\label{rem: counterexample of Nakayama's lemma when M is not finite}
        The finiteness of $M$ is crucial in Nakayama's Lemma.
        For example, let \(\overline{\bbZ}\) be the ring of algebraic integers in \(\overline{\bbQ}\).
        Choose a non-zero prime ideal \(\frakp\) of \(\overline{\bbZ}\).
        Then we have that \(\frakp \overline{\bbZ}_\frakp= \frakp^2 \overline{\bbZ}_\frakp\).
        Indeed, if \(a \in \frakp \overline{\bbZ}_\frakp\), let \(b = \sqrt{a} \in \overline{\bbZ}_\frakp\).
        Then \(b^2 = a \in \frakp \overline{\bbZ}_\frakp\) and whence \(b \in \frakp \overline{\bbZ}_\frakp\) since \(\frakp\) is prime.
        It follows that \(a = b^2 \in \frakp^2 \overline{\bbZ}_\frakp\).
    \end{remark}

    \begin{proposition}[Geometric form of Nakayama's Lemma]\label{prop: geometric form of Nakayama's lemma}
        Let $X = \Spec A$ be an affine scheme, $x\in X$ a closed point and $\calF$ a coherent sheaf on $X$.
        If $a_1,\cdots,a_k \in \calF(X)$ generate $\calF|_x = \calF \ten \rkk(x)$, then there is an open subset $U \subset X$ such that $a_i|_U$ generate $\calF(U)$. 
    \end{proposition}
    \begin{proof}
        \Yang{To be completed.}
    \end{proof}

    \begin{corollary}\label{cor: upper semicontinuity of dimension of restriction of coherent sheaf to fiber}
        Let \(X\) be a scheme and \(\calF\) a coherent sheaf on \(X\).
        Then the function \(x \mapsto \dim_{\rkk(x)} \calF|_x\) is upper semicontinuous.
    \end{corollary}
    \begin{proof}
        \Yang{To be completed.}
    \end{proof}

\subsection{Nullstellensatz}

    Let \(\kk\) be a field and \(\kkk\) be its algebraic closure.

    \begin{theorem}[Noether's Normalization Lemma]\label{thm: Noether's Normalization Lemma}
        Let $A$ be a $\kk$-algebra of finite type.
        Then there is an injection $\kk[T_1,\cdots,T_d] \injmap A$ such that $A$ is finite over $\kk[T_1,\cdots,T_d]$.
    \end{theorem}

    \begin{remark}
        Here $A$ does not need to be integral. 
        For example, 
    \end{remark}

    \begin{theorem}[Hilbert's Nullstellensatz]\label{thm: Nullstellensatz}
        Let $A$ be a \(\kk\)-algebra of finite type.
        \begin{enumerate}
            \item If \(\frakm\) is a maximal ideal of \(A\), then \(A/\frakm\) is a finite extension of \(\kk\).
            \item Suppose that \(\kk\) is algebraically closed and \(A = \kk[x_1,\cdots,x_n]/\fraka\).
                Then there is a bijection between the set of maximal ideals of \(A\) and the set \( \{ (a_1,\cdots,a_n) \in \kk^n \colon f(a_1,\cdots,a_n) = 0, \forall f \in \fraka \} \).
        \end{enumerate}
    \end{theorem}

\section{Finite Algebra and Normality}

Let \(R\) be a ring and \(A\) be an \(R\)-algebra.
We say that \(A\) is \textit{of finite type} over \(R\) if there exists a surjective \(R\)-algebra homomorphism \(R[T_1,\cdots,T_n] \to A\) for some \(n\geq 0\).
We say that \(A\) is finite over \(R\) if it is finite as an \(R\)-module.

\subsection{Finite algebra}

    Let \(A\) be a ring and \(B\) a finite \(A\)-algebra.

    \begin{example}
        Let \(K\) be a number field.
        Then \(O_K\) is a finite \(\bbz\)-algebra.
        \Yang{To be completed.}
    \end{example}

    \begin{lemma}\label{lem: finite inclusion induces surjective morphism}
        Let $A \subset B$ be noetherian rings such that $B$ is finite over $A$.
        Then the induced morphism $\Spec B \to \Spec A$ is surjective.
    \end{lemma}
    \begin{proof}
        For $\frakp \in \Spec A$, let $S:= A-\frakp$ and denote $S^{-1}B$ by $B_\frakp$.
        Then we have $A_\frakp \injmap B_\frakp$ and $B_\frakp$ is finite over $A_\frakp$.
        Let $\frakP B_\frakp$ be a maximal ideal of $B_\frakp$.
        We claim that $\frakP B_\frakp \cap A_\frakp$ is maximal.
        Indeed, consider $A_\frakp/(\frakP \cap A_\frakp) \injmap B_\frakp/\frakP B_\frakp$, the latter is finite over the former.
        This enforces $A_\frakp/(\frakP B_\frakp \cap A_\frakp)$ be a field.
        Hence $\frakP B_\frakp \cap A_\frakp = \frakp A_\frakp$, and then $\frakP \cap A = \frakp$.
    \end{proof}

    \begin{proposition}\label{prop: finite morphisms preserve dimension}
        Let $A \subset B$ be noetherian rings such that $B$ is finite over $A$.
        Then $\dim A = \dim B$.
    \end{proposition}
    \begin{proof}
        If we have a sequence $\frakP_1 \subsetneq \frakP_2$ of prime ideals in $B$, then there exists $f \in \frakP_2 \setminus \frakP_1$.
        Since $B$ is finite over $A$, there exist $a_1,\cdots,a_n \in A$ such that 
        \[ f^n + a_1f^{n-1} + \cdots + a_n = 0.\]

        Then $a_n \in \frakP_2 \cap A$.
        If $a_n \in \frakP_1$, $f^{n-1} + \cdots + a_{n_1} \in \frakP_1$ since $f \notin \frakP_1$.
        Then $a_{n-1} \in \frakP_2$.
        Repeat the process, it will terminate, whence $\frakP_1 \cap A \subsetneq \frakP_2 \cap A$.
        Otherwise, we have $f^n \in a_1 B+ \cdots + a_nB \subset \frakP_1$.

        Conversely, suppose we have $\frakp_1,\frakp_2 \in \Spec A$ with $\frakp_1 \subsetneq \frakp_2$.
        Choose $\frakP_1 \in \Spec B$ such that $\frakP_1 \cap A = \frakp_1$, then we have $A/\frakp_1 \subset B/\frakP_1$.
        Let $\frakP_2$ be the preimage of the prime ideal in $B/\frakP_1$ which is over image of $\frakp_2$ in $A/\frakp_1$.
        Proposition \ref{lem: finite inclusion induces surjective morphism} guarantees that such $\frakP_2$ exists.
        Then we get $\frakP_1 \subsetneq \frakP_2$.
        Repeat this progress, we get $\dim B \geq \dim A$.
    \end{proof}



\Yang{To be completed}
    % Fix a noetherian local ring $(A,\frakm)$ with residue field $\kk$.

    \begin{definition}\label{def: normal domain}
        An integral domain $A$ is called \textit{normal} if it is integrally closed in its field of fractions $\Frac(A)$.
    \end{definition}

    \begin{lemma}\label{lem: integral closed under localization}
        Let $A \subset C$ be rings and $B$ the integral closure of $A$ in $C$, $S$ a multiplicatively closed subset of $A$.
        Then the integral closure of  $S^{-1}A$ in $S^{-1}C$ is $S^{-1}B$.
    \end{lemma}
    \begin{proof}
            For every $b \in B$ and $\forall s \in S$, there exists $a_i \in A$ s.t. 
            \[ 
                b^n + a_1 b^{n-1} + \cdots + a_n = 0.
            \] 
            Then 
            \[ 
                \left( \frac{b}{s} \right)^n + \frac{a_1}{s^1} \left( \frac{b}{s} \right)^{n-1} + \cdots + \frac{a_n}{s^n} = 0.
            \] 
            Hence $b/s$ is integral over $S^{-1}A$, $S^{-1}B$ is integral over $S^{-1}A$.

            If $c/s \in S^{-1}C$ is integral over $S^{-1}A$, then $\exists a_i \in S^{-1}A$ s.t.
            \[ 
                \left( \frac{c}{s} \right)^n + a_1 \left( \frac{c}{s} \right)^{n-1} + \cdots + a_n = 0.
            \]
            Then 
            \[ 
                c^n + a_1 s c^{n-1} + \cdots + a_n s^n = 0 \in S^{-1}C
            \] 
            Then $\exists t \in S$ s.t. 
            \[
                t (c^n + a_1 s c^{n-1} + \cdots + a_n s^n) = 0 \in C.
            \] 
            Then 
            \[ 
                (ct)^n + a_1 s t (ct)^{n-1} + \cdots + a_n s^n t^n  = t^n (c^n + a_1 s c^{n-1} + \cdots + a_n s^n) = 0.
            \] 
            Hence $ct$ is integral over $A$, then $ct \in B$.
            Then $c/s = (ct)/(st) \in S^{-1}B$.
            This completes the proof.
    \end{proof}

    \begin{proposition}\label{prop: normality is a local property}
        Normality is a local property. 
        That is, for an integral domain $A$, TFAE:
        \begin{enumerate}[label=(\roman*)]
            \item $A$ is normal.
            \item For any prime ideal $\frakp \in \Spec A$, the localization $A_\frakp$ is normal.
            \item For any maximal ideal $\frakm \in \mSpec A$, the localization $A_\frakm$ is normal.
        \end{enumerate}
    \end{proposition}
    \begin{proof}
        When $A$ is normal, $A_\frakp$ is normal by Lemma \ref{lem: integral closed under localization}.

        Assume that $A_\frakm$ is normal for every $\frakm \in \mSpec A$.
        If $A$ is not normal, let $\tilde{A}$ be the integral closure of $A$ in $\Frac A$, $\tilde{A}/A$ is a nonzero $A$-module.
        Suppose $\frakp \in \Supp \tilde{A}/A$ and $\frakp \subset \frakm$.
        We have $\tilde{A}_\frakm/A_\frakm = 0$ and $\tilde{A}_\frakp/A_\frakp = (\tilde{A}_\frakm/A_\frakm)_\frakp \neq 0$.
        This is a contradiction.
    \end{proof}

    \begin{proposition}
        Let $A$ be a normal ring.
        Then $A[X]$ is also normal.
    \end{proposition}

    \begin{definition}\label{def: normal of scheme and general ring}
        A scheme $X$ is called \textit{normal} if the local ring $\calo_{X,\xi}$ is normal for any point $\xi \in X$.
        A ring $A$ is called \textit{normal} if $\Spec A$ is normal.
    \end{definition}

    \begin{remark}\label{rmk: total ring of fractions and normality of general reduced ring}
        For a general ring $A$, let $S := A\setminus (\bigcup_{\frakp \in \Ass A} \frakp) = \bigcap_{\frakp \in \Ass A}A\setminus \frakp$.
        Then $S$ is a multiplicative set.
        The localization $S^{-1}A$ is called \textit{the total ring of fractions} of $A$.
        
        Suppose $A$ is reduced and $\Ass A = \{\frakp_1,\cdots,\frakp_n\}$.
        Denote its total ring of fractions by $Q$.
        Note that elements in $Q$ are either unit or zero divisor.
        Hence any maximal ideal $\frakm$ is contained in $\bigcup \frakp_iQ$, whence contained in some $\frakp_i Q$.
        Thus $\frakp_i Q$ are maximal ideals.
        And we have $\bigcap \frakp_i Q = 0$. 
        By the Chinese Remainder Theorem, we have $Q = \prod Q/{\frakp_i}Q = \prod A_{\frakp_i}$.
        
        Let $A$ be a reduced ring with total ring of fractions $Q$.
        Then $A$ is normal iff $A$ is integral closed in $Q$.
        If $A$ is normal, then for every $\frakp \in \Spec A$, $A_\frakp$ is integral.
        Then there is unique minimal prime ideal $\frakp_i \subset \frakp$.
        In particular, any two minimal prime ideal are relatively prime.
        By the Chinese Remainder Theorem, $A = \prod A/\frakp_i$.
        Just need to check $A/\frakp_i$ is integral closed in $A_{\frakp_i}$.
        This is clear by check pointwise.

        Conversely, suppose $A$ is integral closed in $Q$.
        Let $e_i$ be the unit element of $A_{\frakp_i}$.
        It belongs to $A$ since $e_i^2 - e_i = 0$.
        Since $1 = e_1 + \cdots + e_n$ and $e_ie_j = \delta_{ij}$, we have $A = \prod Ae_i$.
        Since $Ae_i$ is integral closed in $A_{\frakp_i}$, it is normal.
        Hence $A$ is normal.
    \end{remark}

    \begin{lemma}\label{lem: normal rings verify R_1 and S_2}
        Let $A$ be a normal ring.
        Then $A$ verifies $(R_1)$ and $(S_2)$.
    \end{lemma}
    \begin{proof}
        Since all properties are local, we can assume $A$ is integral and local.

        For $(S_2)$, by Example \ref{eg: S_2 is equivalent to A/fA has no embedded point}, we only need to show that $\Ass_A A/f$ has no embedded point.
        Let $\frakp = (f:g) = \in \Ass_A A/fA$ and $t:= f/g \in \Frac A$.
        After Replacing $A$ by $A_\frakp$, we can assume that $\frakp$ is maximal.
        By definition, $t^{-1}\frakp \subset A$.
        If $t^{-1}\frakp \subset \frakp$, suppose $\frakp$ is generated by $(x_1,\cdots,x_n)$ and $t^{-1}(x_1,\cdots,x_n)^T = \Phi(x_1,\cdots,x_n)^T$ for $\Phi \in M_n(A)$.
        There is a monic polynomial $\chi(T) \in A[T]$ vanishing $\Phi$.
        Then $\chi(t^{-1}) = 0$ and $t^{-1} \in A$.
        This is impossible by definition of $t$.
        Then $t^{-1}\frakp = A$, and $\frakp = (t)$ is principal. 
        By Krull's Principal Ideal Theorem \ref{thm: Krull's principal ideal theorem}, $\idealht(\frakp) = 1$.

        Now we show that $A$ verifies $(R_1)$.
        Suppose $(A, \frakm)$ is local of dimension $1$.
        Choosing $a \in \frakm$, $A/a$ is of dimension $0$.
        Then by \ref{prop: characteristic of local artinian rings}, $\frakm^n \subset aA$ for some $n\geq 1$.
        Suppose $\frakm^{n-1} \not\subset aA$.
        Choose $b \in \frakm^{n-1} \setminus aA$ and let $t = a/b$.
        By construction, $t^{-1} \notin A$ and $t^{-1}\frakm \subset A$.
        After similar argument, we see that $\frakm = tA$, whence $A$ is regular.
    \end{proof}

    \begin{lemma}\label{lem: normal and regular are equivalent for noetherian rings of dimension 1}
        Let $(A,\frakm)$ be a noetherian local ring of dimension $1$.
        Then $A$ is normal iff $A$ is regular.
    \end{lemma}
    \begin{proof}
        By lemma \ref{lem: normal rings verify R_1 and S_2}, we just need to show that regularity implies normality.

        Let $t \in \frakm\setminus \frakm^2$.
        Since $A$ is regular, $\frakm = (t)$.
        Let $I \subset \frakm$ be an ideal.
        If $I \subset \bigcap_{n} \frakm^n$, then for every $a \in I$, there exists $a_n$ such that $a = a_n t^n$.
        Then we get an ascending chain of ideals $(a_1) \subset (a_2) \subset \cdots$.
        Hence $a=0$ by Nakayama's Lemma.
        Suppose $I$ is not zero.
        Then there is some $n$ such that $I \subset \frakm^n$ and $I \not\subset \frakm^{n+1}$.
        For every $at^n \in I \setminus \frakm^{n+1}$,  $a \notin \frakm$, whence $a$ is a unit in $A$.
        Then $I = (t^n)$.
        Hence $A$ is PID and hence normal.
    \end{proof}

    \begin{proposition}\label{prop: S_2 implies intersection of localization at height 1 prime is A}
        Let $A$ be a noetherian integral domain of dimension $\geq 1$ verifying $(S_2)$.
        Then 
        \[ A = \bigcap_{\frakp \in \Spec A, \idealht(\frakp) = 1} A_\frakp. \]
    \end{proposition}
    \begin{proof}
        Clearly $A \subset \bigcap A_\frakp$.
        Let $t = f/g \in \bigcap A_\frakp$.
        Since $f\in gA_\frakp$ and we have $g A = \bigcap (gA_\frakp \cap A)$, $f \in gA$.
        It follows that $t \in A$.
    \end{proof}

    \begin{theorem}[Serre's criterion for normality]
        Let $X$ be a locally noetherian scheme.
        Then $X$ is normal if and only if it verifies $(R_1)$ and $(S_2)$.
    \end{theorem}
    \begin{proof}
        One direction has been proved in Lemma \ref{lem: normal rings verify R_1 and S_2}.
        Suppose $X$ verifies $(R_1)$ and $(S_2)$.
        Again we can assume $X = \Spec A$ is affine and $A$ is local.
        By Remark \ref{rmk: total ring of fractions and normality of general reduced ring}, we just need to show that $A$ is integral closed in its total ring of fractions $Q$.
        Suppose we have 
        \[ \left(\frac{a}{b}\right)^n + c_1 \left(\frac{a}{b}\right)^{n-1} + \cdots + c_n = 0 \in Q. \]
        Since $A$ verifies $(S_2)$, $b A = \bigcap \nu_\frakp^{-1}(b_\frakp A_\frakp)$.
        So it is sufficient to show that $a_{\frakp} \in b_\frakp A_\frakp$ with $\idealht(\frakp) = 1$.
        Note that $A_\frakp$ is regular and hence normal by Lemma \ref{lem: normal and regular are equivalent for noetherian rings of dimension 1}.
        Then above equation gives us desired result.
    \end{proof}

    